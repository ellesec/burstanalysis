% $Id: damtpleteg.tex,v 1.8 2011/02/14 13:29:33 jp107 Exp $
%
%%%%%%%%%%%%%%%%%%%%%%%%%%%%%%%%%%%%%%%%%%%%%%%%%%%%%%%%%%%%%%%%%%%%%%%%%%%%%%
%
% DAMTP specific documentation can be found at:
%  http://www.damtp.cam.ac.uk/internal/computing/tex/letterhead.html
%
%%%%%%%%%%%%%%%%%%%%%%%%%%%%%%%%%%%%%%%%%%%%%%%%%%%%%%%%%%%%%%%%%%%%%%%%%%%%%%
%%
%% Original version and documentation can be obtained from:
%%    http://www.cl.cam.ac.uk/users/swm11/letters/
%
\documentclass[usebwarms,a4paper,nofold]{damtplet}
%
% Class Options:
%  usebwarms    - uses a black and white version of the Univ. arms
%  usecmfont    - uses computer modern font rather than Sabon (works
%                 better with xdvi (see comment in re font-sizes)
%  usetimesfont - uses times font rather than Sabon
%  dlfold       - mark the place to fold the letter (for
%                 standard university DL window envelopes)
%                 -- this is on by default
%  c5fold       - mark the place to fold the letter (for C5
%                 window envelopes)
%  nofold       - don't mark the place to fold the letter
%
%  e.g.:
%  \documentclass[usecmfont,usebwarms,usetimesfont]{damtplet}
%
% other options are passed to the underlying ``article class''
%
%%%%%%%%%%%%%%%%%%%%%%%%%%%%%%%%%%%%%%%%%%%%%%%%%%%%%%%%%%%%%%%%%%%%%%%%%%%%%%
% Example of a simple letter using Simon's Lab. Letter Class for 2001 as
% modified for use in DAMTP
%

\begin{document}

%%%%%%%%%%%%%%%%%%%%%%%%%%%%%%%%%%%%%%%%%%%%%%%%%%%%%%%%%%%%%%%%%%%%%%%%%%%%%%
% Insert your personal details here
% \fromwhom{name}{qualifications}{title}{email}{phone}{fax}

\fromwhom{Ms~Ellese Cotterill}
        {}
        {Research Student}
        {ec526@cam.ac.uk}
	{+44  1223 337876}
        {+44 1223 765900}
%%\setdate{29$^{\rm th}$ Feb, 2014}
\setsigned{Yours sincerely,}{Ellese~Cotterill}
% \reference{Your ref: 1234/ABCD}
% The institution defaults to DAMTP but can be overridden, note that you
% have to be careful to ensure that the text actually *fits* in the space!
%
% e.g.
%\institution{Faculty of Mathematics}
% or maybe
%\institution{Center for Theoretical\par Cosmology}
%\institution{Center for\par Theoretical Cosmology}
%
% or a longer example 
%\institution{Department of\par Pure Mathematics and\par Mathematical statistics}
%
% If the postcode needs to be different use something like
% \setpostcode{CB3~0WB}
%
% if you want a Cc: list
%\cclist{list of people\\to cc the\\letter to}
%
% if you want to change the default
\setsalutation{Dear Dr. Diamond }
%
% if there are enclosed documents etc
% \enclosures{this\\that\\the other}

% Doras email: angelaki@bcm.edu

\begin{letter}{Dr J. Diamond}{Synaptic Physiology Section, \\ NINDS
Porter Neuroscience Research Center \\
Building 35, Room 3C-1000 \\
35 Convent Drive, MSC 3701  \\
Bethesda, MD 20892-370 \\
United States}
%
% if you want a Title on the letter
% \lettertitle{Example use of \LaTeX\ letter class damtplet.cls}
%
{

 Please find enclosed the manuscript ``A comparison of computational methods for detecting bursts in neuronal spike trains and their application to human stem-cell derived neuronal networks''. 
 \\ \\Accurate identification of patterns of bursting activity is an essential aspect of the analysis of experimental recordings of neuronal activity in a variety of contexts. Despite this, no one widely used method for identifying periods of bursting activity in neuronal spike trains has been adopted in the field. Instead, many methods have been proposed for detecting bursts, however assessment of these methods has generally only been performed by their original authors, using very specific sets of conditions. In this manuscript we present an unbiased assessment of eight previously published burst detection methods. 
 \\ \\We develop a list of desirable properties that a method should ideally possess to accurately detect bursts in a variety of spike trains. We assess each method in regards to these properties using both simulated and experimental data, and show that a number of existing techniques perform poorly at detecting bursts in a range of contexts. Based on this, we provide recommendations for the robust analysis of bursting in neuronal spike trains using existing methods. We also employ a number of the high performing burst detection methods to analyse novel recordings of networks of human induced pluripotent stem cell-derived neurons, and describe the ontogeny of bursting activity in these networks over several months of development.
  \\ \\A significant number of articles that have been previously published in J. Neuroscience, including several in the last twelve months, employ burst detection techniques that were assessed in our study. Since the choice of burst detection technique used to analyse experimental recordings can have implications for the conclusions about the nature of bursting activity in the data, we consider an awareness of the limitations these techniques and the conditions under which they are most suitable to be important. We thus believe that our study would be of relevance to the audience of J. Neuroscience, and your journal would allow the recommendations from our analysis to be accessible to a wide range of both computational and experimental neuroscientists working in this area.
   \\ \\This manuscript has not been published before and all authors have approved the final manuscript.
   \\\\Thank you in advance for your consideration.
}

\end{letter}

\end{document}


%  LocalWords:  retinotopic characterized Neurosci ephrin Willshaw
